\documentclass[acronym,symbols]{fei}

\usepackage[utf8]{inputenc}
\usepackage{verbatim}
\usepackage{outlines}
\usepackage{ulem}
\usepackage{caption}
\usepackage{makecell}
\usepackage{csquotes}
\usepackage{lipsum}  
\usepackage{lscape}

\usepackage{color, colortbl}
	
\definecolor{Green}{rgb}{0.88,1,1}
\definecolor{Gray}{gray}{0.9}

\usepackage{hyperref}
\hypersetup{
    colorlinks=true,
    linkcolor=black,
    citecolor=black,    
    urlcolor=blue,
}


%%%% -- Configuracoes Iniciais%%%%%%%%%%%%%%%

\author{
 Lucas Mateus de  Moraes - RA: 22.220.004-0
}

\title{APLICAÇÃO 1 DE TÉCNICAS DE PROCESSAMENTO DE LINGUAGEM NATURAL PARA AVALIAÇÃO AUTOMÁTICA DE QUESTÕES DISSERTATIVAS}

% comando para inserção de subfloats do tipo figure, usado aqui no template
% remova este comando se for usar o pacote subfig
% não recomendo o pacote subcaption
\newsubfloat{figure}

\addbibresource{referencias.bib}

\begin{document}
\maketitle 

\begin{resumo}

Este trabalho propõe uma abordagem para a avaliação automática de respostas dissertativas em ambientes educacionais. A métrica proposta pelo presente trabalho combina a presença de palavras-chave, a frequência de sentidos de palavras, a análise da sintaxe do texto e a distância semântica através do cosseno para oferecer uma solução que simplifica o processo de correção manual. Utilizando o \textit{DataSet ASQA} do \textit{Google} e dados de respostas do ENADE, a pesquisa busca validar a eficácia da técnica proposta considerando fatores relevantes para processamento de linguagem natural.

A proposta possui seus desafios de pesquisa, estudo e desenvolvimento, porém, não possui problemas significativos em relação a obter bases de dados ou lidar com dados sensíveis. Além de que, possui uma valor de contribuição valioso para a presente instituição de ensino e outras universidades, também sendo capaz de contribuir com o ensino de forma geral, pois, o uso de inteligência artificial (\textit{AI}) e processamento de linguagem natural (\textit{NLP}) aplicados a educação ainda é um tópico pouco explorado e de grande relevância, podendo assim, tornar a universidade e o presente trabalho uma possível referência no assunto, abrindo espaço para aplicações reais caso seu desenvolvimento futuro seja bem sucedido.

\palavraschave{Semantic Similarity; Paraphrase Detection; Natural Language Processing; Evaluation of Descriptive Answers}

\end{resumo}

\begin{abstract}

This article proposes an innovative approach for the automated evaluation of essay-type responses in educational environments. In contrast to the preference for objective questions, the proposed metric combines keyword presence, word sense frequency, syntax analysis, and semantic distance using cosine similarity to offer a solution that streamlines the manual correction process. Utilizing Google's ASQA DataSet and data from ENADE's exam, the research aims to validate the effectiveness of the technique, considering relevant factors for natural language processing.

The proposal presents challenges in research, study, and development; however, it does not encounter significant issues regarding obtaining databases or handling sensitive data. Moreover, it holds valuable contribution potential for the current educational institution and other universities. It is also capable of contribute to education in general, as the use of artificial intelligence (\textit{AI}) and natural language processing (\textit{NLP}) applied to education is still a relatively unexplored and highly relevant topic. Therefore, the university and the present work could become a reference in the field, opening space for real applications if its future development is succeeds.

\keywords{Semantic Similarity; Paraphrase Detection; Natural Language Processing; Evaluation of Descriptive Answers}

\end{abstract}

\listoffigures
\listoftables

\tableofcontents

\chapter{Introdução}

Na atualidade das instituições de ensino, uma variedade de métodos de avaliação são empregados para mensurar o aprendizado dos alunos, destacando-se entre eles as questões dissertativas, nas quais os alunos devem fornecer suas respostas de maneira textual. Esse método proporciona ao professor uma compreensão mais aprofundada da linha de raciocínio do aluno durante a correção, possibilitando assim, uma avaliação mais precisa do nível de aprendizado alcançado e um melhor acompanhamento da evolução do aluno ao longo do tempo \cite{artigoPorcentagemQuestoes}.

Embora a precisão da avaliação seja uma vantagem desse método, o mesmo exige maior tempo de leitura, análise e compreensão de cada questão por parte do professor, uma vez que ele precisará analisar integralmente o conteúdo dos textos fornecidos como resposta pelos alunos. 

Tal dificuldade pode levar o docente a preferir questões objetivas, que, por possuírem um gabarito, demandam um tempo menor de correção. Entretanto, é importante ressaltar que as questões objetivas não atingem o mesmo nível de precisão na avaliação em comparação com as questões dissertativas, uma vez que possibilitam que o aluno escolha uma alternativa de maneira aleatória e possa obter a resposta correta mesmo sem ter nenhum conhecimento da mesma, ao contrário das questões dissertativas, em que essa possibilidade não existe.

A utilização de avaliações dissertativas é utilizado em apenas 30\% das formas de avaliação aplicadas aos alunos, conforme demonstrado em um estudo realizado por % pelas pesquisadoras Katya Luciane de Oliveira, Mestre em Psicologia pela Universidade São Francisco, e Acácia Aparecida Angeli dos Santos, Doutora em Psicologia Escolar e do Desenvolvimento Humano pela USP.
\textcite{artigoPorcentagemQuestoes}. Este estudo aborda as vantagens das avaliações dissertativas e sua maior adequação para a avaliação do desempenho e acompanhamento do progresso dos alunos no processo de aprendizado. Considerando esse contexto, seria benéfico para as instituições de ensino adotar mais frequentemente avaliações dissertativas \cite{artigoPorcentagemQuestoes}. No entanto, como mencionado anteriormente, um dos principais impeditivos da adoção desse método seria o aumento na carga de trabalho dos professores responsáveis pelas correções. 

Por conta disso, o presente trabalho propõe uma abordagem para realizar avaliação automática das respostas dissertativas dos alunos, comparando-as com uma resposta padrão fornecida por um especialista da área como modelo do conhecimento esperado para aquela questão.

De forma geral, o algoritmo desenvolvido neste trabalho avalia o "grau de similaridade" entre as respostas das questões levando em consideração propriedades da parte léxica, da parte de sintaxe e da parte semântica dos textos. No fim, o algoritmo utiliza uma métrica de média ponderada para mensurar a similaridade entre as respostas e as referências de professores ou especialistas e gerar uma avaliação automática das respostas fornecidas para determinadas questões. Para que essa avaliação fosse efetivamente realizada, alguns fatores eram extraídos do texto, os quais podemos considerar como elementos relevantes na composição do texto de uma resposta. Esses fatores foram:

\begin{itemize}
    \item Similaridade semântica
    \item Frequência de termos
    \item Distância de Levenshtein
\end{itemize}

% presença de palavas chaves, número de modificações para 
% transformar uma resposta na outra, ordem das palavras, similaridade de texto, grau de parafraseamento, entre outros.

%Como um desses fatores podemos considerar a presença de palavras-chave que devem constar na resposta, podendo ser especificadas pelo professor, cada uma podendo ter um peso na avaliação da similaridade semântica. Outros fatores, como a quantidade de modificações necessárias para transformar a resposta do aluno na resposta original %(o que pode ser chamado de grau de parafraseamento), , a distância entre os cossenos dos dois textos (que fornece um grau de proximidade entre o conteúdo de ambos) e a ordem das palavras (que diz respeito à sintaxe do texto) também podem ser considerados, cada um com pesos apropriados.  Além disso, pretende-se investigar se toda resposta correta é, em algum grau, uma paráfrase da resposta padrão, em que o sentido semântico é preservado.
%Porém, levando em conta que a correção de respostas dissertativas é algo 

%Sendo assim, o objetivo é propor uma métrica que avalie as respostas fornecidas em comparação com uma resposta correta definida como modelo para uma determinada questão, visando que essa métrica possa ser usada para avaliações da forma mais ampla possível.

\section{Objetivo}

O objetivo final do trabalho foi desenvolver e implementar o algoritmo para uma abordagem automatizada de avaliação para respostas dissertativas. A proposta teve como alvo simplificar o processo de correção manual dessas respostas, promovendo uma análise dos fatores extraídos do texto e sua relação com as avaliações geradas. As metas planejadas no trabalho foram especificadas nos seguintes tópicos:

\begin{itemize}
\item Desenvolver um algoritmo para mensurar a similaridade semântica entre respostas dissertativas e uma resposta padrão de um professor como referência.
\item Considerar os fatores de similaridade semântica, frequência de termos e distância de Levenshtein.
\item Validar a eficácia do algoritmo comparando-o com dados de avaliações já corrigidas.
\item Como última meta, planejada para o caso das anteriores serem alcançadas, foi feito um protótipo para testes práticos do algoritmo visando elucidar os conceitos abordados no trabalho.
\end{itemize}

% \item Considerar fatores como a presença de palavras-chave, quantidade de repetições de palavras, grau de parafraseamento, distância entre cossenos dos textos e ordem das palavras na avaliação automática.

\section{Questões de Pesquisa}

O tema abordado levanta importantes questões de pesquisa, nas quais o presente trabalho buscou responder questões tais como:

\begin{enumerate}
\item Quais parâmetros podemos extrair como fatores do texto que devem ser considerados como componentes relevantes para pontuar a similaridade semântica entre as respostas e as referências?
\item Esses parâmetros podem definir uma pontuação que funcione de maneira geral quando aplicada a casos práticos de correções dissertativas?
\item Como os fatores devem ser ponderados dentro de uma métrica para as avaliações?
\item Como a acurácia da métrica para as avaliações varia conforme os eventuais pesos, dados para treino e dados para teste dos fatores também variam?
\end{enumerate}

% \item Toda resposta correta pode ser considerada, em algum grau, uma paráfrase de uma resposta padrão?
% \item A partir de qual ponto pode-se dizer que o grau de paráfrase entre os textos da resposta de um aluno e a resposta padrão indica corretude? Ou, em vez de uma saída binária, o resultado deve ser representado por uma escala variável?

\newpage
\chapter{Conceitos}

Nesta seção serão apresentados alguns conceitos fundamentais para o entendimento da porposta desse projeto. Serão abordados conceitos de processamento de lingugagem natural, formas de representação de texto em formato númerico e metricas para comparação de textos e avalialção de desempenho. Assim livros como \textit{Introduction to natural language processing} \cite{NLP} e \textit{Machine Learning: An Algorithmic Perspective} \cite{ML}, são importantes fontes para o aprofundamento nesses tópicos.
%Para o bom desenvolvimento do projeto é necessário a revisão de assuntos basilares para \textit{NLP} e um aprofundamento nos principais tópicos relacionados as técnicas utilizadas na área. 

\section{Processamento de linguagem natural}

O Processamento de Linguagem Natural (PLN) refere-se à aplicação de técnicas computacionais para a interpretação e manipulação de linguagem humana. Envolve o desenvolvimento de algoritmos e modelos que capacitam computadores a compreender, analisar e gerar texto de maneira semelhante ao entendimento humano.

\section{Representação de textos}

A Representação de Textos é crucial para permitir que algoritmos compreendam palavras e documentos. Duas técnicas comuns são TFIDF (Term Frequency-Inverse Document Frequency) e Word Embedding. O TFIDF avalia a importância de uma palavra em um documento, enquanto o Word Embedding mapeia palavras em vetores contínuos, capturando relações semânticas.

%\subsection{TFIDF}
%O TFIDF é uma técnica que atribui pesos a palavras com base em sua frequência no documento e em todo o conjunto de documentos. Palavras frequentes em um documento, mas raras no conjunto de documentos, recebem pontuações mais altas, destacando sua relevância no contexto do documento específico.

\subsection{Word embedding}
O Word Embedding é uma técnica que mapeia palavras em vetores de números reais, capturando relações semânticas e contextuais. Essa representação densa permite que algoritmos de processamento de linguagem natural compreendam a similaridade e a semântica entre palavras. Considere as palavras "rei" e "rainha." Se estiverem bem representadas por embeddings, a subtração dos vetores "rei" e "homem" deve ser aproximadamente igual à subtração dos vetores "rainha" e "mulher," refletindo a relação semântica de gênero.

\section{Similaridade de textos}
A Similaridade de Textos é fundamental para comparar documentos ou palavras. Diversas métricas são empregadas, como Palavras-Chave, Frequência de Palavras, Distância de Jaccard, Distância de Cosseno, Distância Euclidiana e Modelos de Linguagem.

\subsection{Palavras-chave}
A similaridade pode ser avaliada considerando as palavras-chave mais relevantes em documentos. A sobreposição ou relevância compartilhada entre essas palavras indica o grau de similaridade.Considere dois documentos sobre inteligência artificial. Se ambos compartilharem palavras-chave como "aprendizado de máquina", "algoritmos" e "processamento de linguagem natural," é provável que sejam semanticamente similares.

\subsection{Frequência de Sentidos de Palavras}

A frequência de sentidos de lalavras (SFD) é uma métrica que avalia a distribuição de frequência dos diferentes sentidos ou significados associados a uma palavra ao longo de um conjunto de documentos. Em outras palavras, visa entender como a polissemia (múltiplos significados) de uma palavra se distribui em contextos específicos. Essa métrica é relevante para a detecção de mudanças semânticas em textos ao longo do tempo. Considerando a palavra "bateria" por exemplo que pode ser um instrumento musical ou um dispositivo eletrônico para armazenar energia elétrica. Se, ao longo do tempo, a frequência de uso de "bateria" em contextos relacionados a música diminuir enquanto o uso em contextos de armazenamento de energia aumentar, a SFD refletirá essa mudança semântica

\subsection{Distância de Jaccard}
A Distância de Jaccard avalia a similaridade entre conjuntos, medindo a proporção de elementos comuns entre dois conjuntos. Para texto, representa a sobreposição de palavras entre dois documentos. Considere dois conjuntos de palavras em dois documentos. Se o Conjunto A contiver as palavras \{a, b, c\} e o Conjunto B as palavras \{b, c, d\}, a Distância de Jaccard seria de 50\% de similaridade, como na demonstrado na Equação 1. 
\begin{equation}
\frac{|A \cap B|}{|A \cup B|} = \frac{2}{4} = 0.5
\end{equation}

\subsection{Distância de Cosseno}
A Distância de Cosseno mede o ângulo entre dois vetores de palavras, representando a similaridade direcional entre documentos. Quanto menor o ângulo, maior a similaridade. Considere dois vetores de palavras representando documentos. Se esses vetores apontarem na mesma direção, a distância de cosseno será próxima de zero, indicando alta similaridade. Se apontarem em direções opostas, a distância será próxima de 1, indicando baixa similaridade.

\subsection{Distância Euclidiana}
A Distância Euclidiana calcula a distância geométrica entre pontos em um espaço vetorial. Em texto, representa a dissimilaridade entre as distribuições de palavras. Considere dois documentos representados como pontos em um espaço vetorial. Se os pontos (representando os documentos) estiverem próximos no espaço, a distância euclidiana será pequena, indicando alta similaridade. Se estiverem distantes, a distância será grande, indicando baixa similaridade.

\subsection{Distância Jensen-Shannon}
A Distância Jensen-Shannon (JSD) é uma medida de divergência estatística entre duas distribuições de probabilidade que é utilizada em processamento de linguagem natural para avaliar a similaridade entre textos com base na distribuição de frequência das palavras. O cálculo da JSD envolve a criação de uma distribuição média ponderada e o uso da entropia de Kullback-Leibler. Ela considera não apenas a presença ou ausência de palavras, mas também a probabilidade de ocorrência dessas palavras nos textos. Quanto menor a distância obtida, maior é a similaridade semântica entre os textos.

\subsection{Language Models}
Os Modelos de Linguagem, como os de Parafraseamento, buscam entender a similaridade semântica entre frases ou documentos, indo além da análise baseada em palavras. Supondo que um modelo de linguagem deve prever palavras em frases. Na frase "O gato está na", o modelo de linguagem pode prever as palavras "casa", "árvore" e rua por exemplo. Com base em um conjunto de dados de treinamento a probabilidade da palavra "casa" pode ser maior.

\subsubsection{Parafraseamento}
Modelos de Parafraseamento são específicos para avaliar a similaridade entre frases ou documentos que expressam a mesma ideia de maneira diferente. Esses modelos buscam capturar nuances semânticas e estruturais, identificando relações de equivalência entre diferentes formulações. As frases "O clima estava agradável para um passeio no parque" e "Por conta do clima agradável, o parque seria um bom passeio" devem ser reconhecidas por um modelo de parafraseamento já que ambas as frases têm uma intenção semelhante, apesar das diferenças na escrita

\section{Métricas de avaliação}
As Métricas de Avaliação quantificam o desempenho de modelos de processamento de linguagem natural. A acurácia é uma medida fundamental, representando a proporção de predições corretas em relação ao total. Outras métricas, como precisão, revocação e F1-Score, oferecem insights adicionais sobre o desempenho do modelo em diferentes aspectos da classificação ou similaridade.

\subsection{Acurácia}
A acurácia é uma métrica fundamental de avaliação, comumente usada para medir o desempenho geral de modelos de processamento de linguagem natural. Representa a proporção de predições corretas em relação ao total de predições. Embora seja uma medida direta, a Acurácia pode ser limitada em cenários desbalanceados, sendo complementada por métricas adicionais, como precisão, revocação e F1-Score, para avaliação mais abrangente do desempenho do modelo. No entanto, em cenários desbalanceados, a acurácia pode ser limitada. Para avaliação mais abrangente, métricas adicionais, como BLEU (Bilingual Evaluation Understudy) e ROUGE (Recall-Oriented Understudy for Gisting Evaluation), podem ser utilizadas.

\subsubsection{BLEU (Bilingual Evaluation Understudy)}
O BLEU é comumente usado para avaliar a qualidade de traduções automáticas em tarefas de processamento de linguagem natural. Ele calcula a sobreposição de palavras entre a tradução gerada pelo modelo e a tradução de referência. Quanto mais sobreposição, maior é o escore BLEU.

\subsubsection{ROUGE (Recall-Oriented Understudy for Gisting Evaluation)}
O ROUGE é empregado para avaliar a qualidade de resumos automáticos, focando na recordação (revocação) das palavras-chave. Ele mede a sobreposição de n-gramas (sequências contínuas de n palavras) entre o resumo gerado e o resumo de referência. Maior sobreposição resulta em um escore ROUGE mais alto.

%\section*{Técnicas de classificação/similaridade de texto}

%As principais técnicas utilizadas nos artigos selecionados foram:

%\begin{itemize}
%    \item Artigo \cite{DeterminingDegreeRelevanceReviewsUsingGraphBasedTextRepresentation}: O algoritmo \textit{K-nearest neighbor classification} é usado para construção de um modelo. A representação de texto é baseada em um grafo para identificar igualdade de sintaxe entre textos. As métricas baseadas em \textit{string} incorporam conceitos de paráfrase e plágio para identificar a similaridade de textos. Métricas como \textit{, }
%
%    \item Artigo 
%\end{itemize}

\newpage
\chapter{Revisão Bibliográfica}

Para realização da revisão bibliográfica foram utilizados as ferramentas de buscas de artigos científicos do \textit{Google Scholar} (\url{https://scholar.google.com/}), \textit{IEEE Xplore} (\url{https://ieeexplore.ieee.org/Xplore/home.jsp}), \textit{ScienceDirect} (\url{https://www.sciencedirect.com/}) e \textit{Semantic Scholar} (\url{https://www.semanticscholar.org/}).

Como palavras-chaves na busca foram utilizados termos em inglês, sendo eles, \textit{"semantic similarity between texts"}, \textit{"measure degree of paraphrase"}, \textit{"paraphrase detection"}, \textit{"natural language processing"}, \textit{"measure semantic similarity between answers"} e \textit{"evaluation of descriptve answers"}. Os termos que trouxeram os melhores resultados e estavam presentes nos melhores artigos selecionados foram \textit{"semantic similarity"}, \textit{"natural language processing"} e \textit{"evaluation of descriptve answers"}.

Inicialmente foram selecionados 26 artigos que poderiam ser relevantes para o presente trabalho com base nos temas, a sumarização dos artigos pode ser vista na Tabela \ref{table:1} em que os artigos selecionados no final estão destacados na cor cinza.

% Please add the following required packages to your document preamble:
% \usepackage{booktabs}
% \usepackage{graphicx}
\begin{table}[htb!]
\centering
\resizebox{\columnwidth}{!}{%
\begin{tabular}{@{}lll@{}}
\toprule
\textbf{Título} & Fonte & Referência \\ \midrule
Determining Degree of Relevance of Reviews Using a Graph-Based Text Representation & IEEE & \cite{DeterminingDegreeRelevanceReviewsUsingGraphBasedTextRepresentation} \\
A Chinese text paraphrase detection method based on dependency tree & IEEE & \cite{ChineseParaphraseDetectionBasedDependencyTree} \\
Nowhere to Hide: Finding Plagiarized Documents Based on Sentence Similarity & IEEE & \cite{FindingPlagiarizedDocumentsBasedSentenceSimilarity} \\
Enhanced Text Matching Based on Semantic Transformation & IEEE & \cite{EnhancedTextMatchingBasedSemanticTransformation} \\
Semantic similarity based assessment of descriptive type answers & IEEE & \cite{SemanticSimilarityBasedAssessmentDescriptiveTypeAnswers} \\
A comparative analysis of various approaches for automated assessment of descriptive answers & IEEE & \cite{comparativeAnalysisVariousApproachesAutomatedAssessmentDescriptiveAnswers} \\
A reliable approach to automatic assessment of short answer free responses & SSL & \cite{ReliableApproachAutomaticAssessmentShortAnswerFreeResponses}  \\
A Descriptive Answer Evaluation System Using Cosine Similarity Technique & IEEE & \cite{DescriptiveAnswerEvaluationSystemUsingCosineSimilarityTechnique} \\
An Intelligent System for Evaluation of Descriptive Answers & IEEE & \cite{IntelligentSystemEvaluationDescriptiveAnswersFuzzyGraphs} \\
Application Research of Similarity Algorithm in the Design of English Intelligent Question Answering System & IEEE & \cite{ApplicationResearchSimilarityAlgorithmDesignEnglishIntelligentQuestionAnsweringSystem} \\
Near duplicate text detection using graph depiction & IEEE & \cite{NearDuplicateTextDetectionKendalRankCorrelationModelTVM} \\
Recognition of Parallelism Sentence Based on Recurrent Neural Network & IEEE & \cite{RecognitionParallelismSentenceBasedRecurrentNeuralNetwork} \\
LSGC: An Interactive Text Matching Model Combined with Enhanced Encoding & IEEE & \cite{LSGCInteractiveTextMatchingModelEnhancedEncoding} \\
A software system for determining the semantic similarity of short texts in Serbian & IEEE & \cite{SoftwareSystemSemanticSimilarityShortTextSerbian} \\
Arabic Semantic Textual Similarity Identification based on Convolutional Gated Recurrent Units & IEEE & \cite{ArabicSemanticTextualSimilarityConvolutionalGatedRecurrentUnits} \\
A Chinese text paraphrase detection method based on dependency tree & IEEE & \cite{ChineseParaphraseDetectionBasedDependencyTree} \\
Using paraphrases to improve tweet classification: Comparing WordNet and word embedding approaches & IEEE & \cite{UsingParaphrasesTweetClassificationComparingWordNetAndWordEmbedding} \\
\rowcolor{Gray} SemEval-2020 Task 1: Unsupervised Lexical Semantic Change Detection & SSL & \cite{UnsupervisedLexicalSemanticChangeDetection} \\
SemEval-2017 Task 1: Semantic Textual Similarity Multilingual and Crosslingual Focused Evaluation & SSL & \cite{SemanticTextualSimilarityMultilingualCrosslingualFocusedEvaluation} \\
\rowcolor{Gray} Use of Syntactic Similarity Based Similarity Matrix for Evaluating Descriptive Answer & IEEE & \cite{SyntacticSimilarityBasedSimilarityMatrixForEvaluatingDescriptiveAnswer} \\
\rowcolor{Gray} Chapter 16 - Semantic similarity–based descriptive answer evaluation & SCD & \cite{SemanticSimilarityBasedDescriptiveAnswerEvaluation} \\
\rowcolor{Gray} A Study of Automated Evaluation of Student’s Examination Paper using Machine Learning Techniques & IEEE & \cite{StudyAutomatedEvaluationStudentsExaminationPaperMachineLearningTechniques} \\
Online Examination with short text matching & IEEE & \cite{OnlineExaminationWithShortTextMatching} \\
Towards Automated Evaluation of Handwritten Assessments & IEEE & \cite{TowardsAutomatedEvaluationHandwrittenAssessments} \\
Automatic Short Answer Grading (ASAG) using Attention-Based Deep Learning MODEL & IEEE & \cite{AutomaticShortAnswerGrading-ASAG-usingAttentionBasedDeepLearningMODEL} \\
Semantic similarity–based descriptive answer evaluation & SSL & \cite{SemanticSimilarityBasedDescriptiveAnswerEvaluation} \\ \bottomrule
\end{tabular}%
}
\caption{Tabela de revisão bibliográfica sumarizada.}
\label{table:1}
\end{table}


%\begin{figure}
%    \centering
%    \includegraphics[width=\textwidth]{imgs/rev.png}
%    \caption{Figura da tabela de revisão bibliográfica sumarizada.}
%    \label{fig:1}
%\end{figure}

Após uma filtragem com base nos resumos e palavras-chaves, tendo como critério, a semelhança dos termos e a similaridade de outros artigos em comparação com a proposta do presente artigo, foram selecionados quatro artigos como base para a revisão bibliográfica como consta na Tabela \ref{table:2}, para leitura integral de seu conteúdo.

\begin{table}[htb!]
\resizebox{\columnwidth}{!}{%
\begin{tabular}{@{}lrrrr@{}}
\toprule
\textbf{Base} &
  \multicolumn{1}{l}{Total encontrados} &
  \multicolumn{1}{l}{Após remoção dos Duplicados} &
  \multicolumn{1}{l}{Após análise do resumo} \\ \midrule 
IEEE             & 21 & 21 & 2 \\
Science Direct   & 1  & 1  & 1 \\
Semantic Scholar & 4  & 3  & 1 \\ \bottomrule
\end{tabular}%
}
\caption{Tabela do funil de leitura.}
\label{table:2}
\end{table}

\newpage

O artigo proposto por \cite{UnsupervisedLexicalSemanticChangeDetection}, faz o uso de \textit{embeddings} de tipo (\textit{type embeddings}) e \textit{embeddings} contextualizados (\textit{token embeddings}) para representar as palavras. Primeiro, é introduzida a distribuição de frequência de sentido (SFD), e a detecção de mudança binária é definida em termos de limiares de frequência. Após isso, a distância de Jensen-Shannon (JSD) entre as distribuições normalizadas de frequência é utilizada para medir a mudança efetuada.
    
No artigo proposto por \cite{SyntacticSimilarityBasedSimilarityMatrixForEvaluatingDescriptiveAnswer}, podemos destacar que o trabalho utiliza a técnica de Análise Semântica Latente (LSA), que é comumente usada para determinar a similaridade de documentos, mas ressalta suas limitações em documentos curtos. O artigo destaca a ausência de abordagens anteriores que se concentrem na avaliação automática de respostas descritivas usando vetores de ordem de palavras. O método proposto utiliza uma matriz de similaridade entre vetores de ordem de palavras para avaliar respostas descritivas. A similaridade entre os vetores é calculada por meio de uma métrica de similaridade sintática, ou seja baseada na ordem das palavras. Os resultados indicam que a abordagem baseada em ordem de palavras é promissora para a avaliação automática de respostas descritivas. A matriz de similaridade é apresentada como uma ferramenta eficaz para computar as notas de cada pergunta.
    
Na proposta do artigo escrito pelos autores \cite{SemanticSimilarityBasedDescriptiveAnswerEvaluation}, pode-se destacar que a pesquisa faz uso de Processamento de Linguagem Natural (NLP) para automatizar o processo de avaliação, especialmente a similaridade de cosseno e índices de similaridade, são empregadas para atribuir notas às respostas.
    
Os autores \cite{StudyAutomatedEvaluationStudentsExaminationPaperMachineLearningTechniques} incorporam uma abordagem que emprega ferramentas de Reconhecimento Óptico de Caracteres (OCR) para extrair texto de respostas manuscritas digitalizadas. A ênfase principal, no entanto, recai sobre o emprego de técnicas avançadas de processamento de linguagem natural para aprimorar a avaliação. O estudo destaca a importância de etapas como a tokenização, remoção de stop words e verificação de sinônimos e antônimos no pré-processamento das respostas. Além disso, aborda a criação de modelos semânticos e o cálculo de similaridade sem mencionar explicitamente as métricas de Machine Learning utilizadas.



No contexto da revisão bibliográfica, alguns conceitos importantes foram retirados dos artigos selecionados, para contribuir com o presente trabalho. Dentre esses, destacam-se a análise da similaridade semântica com cossenos, a consideração da frequência de Sentidos de Palavras (SFD) e a utilização de vetores de ordem de palavras como elementos-chave.

A análise de similaridade semântica com cossenos é uma técnica importante, conforme evidenciado nos artigos revisados \cite{SemanticSimilarityBasedDescriptiveAnswerEvaluation}. Essa abordagem é frequentemente empregada para medir a proximidade semântica entre textos.

O uso da Frequência de Sentidos de Palavras (SFD) em um dos artigos indica a importância específica da distribuição de frequência das palavras. Esse conceito pode contribuir para o trabalho, sendo relevante para aspectos da sintaxe e da semântica do texto.

A utilização de vetores de ordem de palavras, como abordado em um dos artigos, resalta a importância da ordem das palavras na avaliação de respostas descritivas. Essa técnica pode superar limitações associadas à Análise Semântica Latente (LSA) em documentos curtos, oferecendo uma abordagem promissora para a avaliação automática.

Em síntese, a revisão bibliográfica gerou a necessidade de explicitar conceitos fundamentais a análise de similaridade semântica com cossenos, a distribuição de frequência de sentidos de palavras e da exploração de vetores de ordem de palavras como pontos relevantes nos estudos revisados.

\newpage
\chapter{Metodologia}

Nesta seção, descrevemos a metodologia adotada para desenvolver e testar o algoritmo de avaliação automática de respostas dissertativas. O processo abrange desde a formatação dos dados até a comparação dos resultados gerados pelo algoritmo com as avaliações realizadas pelos docentes. De forma visual o processo geral pode ser visto na Figura \ref{figure:29}

\begin{figure}[h!]
\centering
\includegraphics[width=0.8\textwidth]{img/tccFlux.png}
\caption{Diagrama do fluxo completo do algoritmo}\label{figure:29}
\end{figure}

\FloatBarrier

Na metodologia, de forma geral, foram retirados dos textos disponíveis nas bases de dados os três fatores que são usados regressão linear e posteriormente nos testes de geração de avaliações. Esses fatores são:

\begin{itemize} 
  \item \textbf{As propriedades léxicas do texto:} O primeiro fator será levado em consideração fazendo uso das frequências de termos de cada texto. 
  \item \textbf{As propriedades da sintaxe do texto:} O segundo fator será levado em consideração fazendo uso da distância de Levenshtein. 
  \item \textbf{As propriedades da semântica do texto:} O terceiro fator será levado em consideração fazendo uso dos valores de similaridade semântica fornecidos pelos embeddings gerados pelo modelo BERT. 
\end{itemize}

A métrica utilizada na metodologia pode ser matemáticamente descrita como o exemplo da Equação 1, que contém a equação de uma média ponderada.
\begin{equation}
Métrica = \frac{fator1 \times peso_{1} + fator2 \times peso_{2} + fator3 \times peso_{3}}{\sum_{i=1}^{3}peso_{i}}
\label{eq:1}
\end{equation}

\section{Pipeline de execução do algoritmo}

O pipeline de execução do algoritmo compreende várias etapas, desde a preparação dos dados até a análise dos resultados. As etapas podem ser detalhadas da seguinte forma:

\subsection{Formatação dos Dados}

A primeira etapa envolve a formatação dos dados oriundos de diferentes bases, convertendo-os em um modelo comum. 

Uma das principais dificuldades encontradas inicialmente no presente trabalho foi o tratamento de dados provenientes de múltiplas bases em diferentes idiomas, cada uma com formatos e estruturas distintas. 

Isso exigiu a implementação de algoritmos de pré-processamento e normalização de texto para garantir a consistência dos dados antes da análise. Isso foi fundamental para desenvolver abordagens eficazes de tratamento de dados em inglês, espanhol e português.

Isso garante a padronização dos dados, facilitando o processamento subsequente. A estrutura de dados é definida como demonstrado no diagrama da Figura \ref{figure:28}, onde cada entrada é tipada e preparada para a análise, no fim essas informações são salvas em arquivo no formato \textit{.json} para facilitar a leitura com auxílio de bibliotecas prontas na linguagem de programação \textit{Python}.

\begin{figure}[h!]
\centering
\includegraphics[width=0.6\textwidth]{img/DataClassModelsTCC.png}
\caption{Modelo de padronização dos dados representados como classes}\label{figure:28}
\end{figure}

\FloatBarrier

\subsection{Extração e Normalização dos Fatores}

Após a formatação, os dados são processados pelo algoritmo para a extração de três fatores principais que influenciam a avaliação. Primeiramente, é feito o cálculo da distância de Levenshtein entre o texto da resposta de referência do professor e a resposta de um aluno disponível na base de dados, retornando um valor para avaliar a proximidade da sintaxe dos dois textos. Em seguida, o cálculo da frequência de termos usando \textit{Count Vectorization} é feito para ambos os textos, e a comparação entre os vetores gerados para eles é feita usando a distância de cosseno, gerando um valor para avaliar a semelhança léxica dos textos.

Depois disso, utilizando o modelo BERT, os embeddings dos textos são calculados. O BERT utiliza uma técnica chamada "WordPiece tokenization" para dividir o texto em pedaços menores, chamados de "tokens". Cada token é então convertido em um vetor de números reais através de uma camada de embeddings. Esses embeddings capturam o contexto e o significado das palavras no texto. Posteriormente, os embeddings de cada texto são organizados em matrizes e manipulados para calcular a similaridade entre eles utilizando a distância de cosseno, o que resulta em um valor que representa a similaridade semântica entre os dois textos.

Esse processo é repetido algumas vezes apenas para o fator semântico, pois existem modelos diferentes do BERT em cada idioma e versões diferentes (\textit{Base} e \textit{Large}). É necessário utilizar os dados gerados por cada um dos modelos para avaliar o desempenho do algoritmo com cada um deles.

No fim, todos os fatores são normalizados com base nos valores mínimos e máximos das notas fornecidas pelos professores. A normalização é crucial para garantir que os fatores se situem dentro da mesma escala na qual as notas devem ser avaliadas.

\subsection{Regressão Linear}

Com os fatores normalizados, a próxima etapa é a aplicação de uma regressão linear. A regressão é treinada comparando os fatores extraídos com as notas das avaliações dos professores. Isso permite determinar os pesos específicos para cada fator. Os pesos obtidos são essenciais para a fase subsequente, onde serão usados para gerar novas avaliações. Além dos pesos, com a regressão também é possível obter os valores do EQM (\textit{Mean Squared Error} ou MSE) e do EMA (\textit{ Mean Absolute Error} ou MAE).

\subsection{Geração de Avaliações}

Utilizando os pesos determinados pela regressão linear, o algoritmo gera avaliações automáticas para um conjunto de dados separados. Esses dados são as respotas dos alunos em comparadas com as referências dos professsores, porém nesse momento os três fatores são retirados dos textos e logo em seguida uma avaliação de nota é gerando fazendo o cálculo da média ponderada, para isso são usados, em conjunto com os valores de cada fator, os valores de cada peso que foram determinados anteriormente na regressão linear. Após isso, ainda nesta etapa, os valores das avaliações geradas pelo algoritmo são guardados para serem comparados com as avaliações feitas pelos próprios professores para as mesmas respotas de alunos disponíveis nas bases de dados, assim podermos constatar se há semelhança entre os valores de ambas as avaliações, do algoritmo e dos professores.

\subsection{Comparação e Cálculo de Acurácia}

Finalmente, as avaliações geradas pelo algoritmo são comparadas com as notas originais fornecidas pelos professores na base de dados. Essas comparações são feitas dividindo o conjunto de dados em diferentes quantias de valores. Primeiramente, 60\% dos dados são usados para o treino da regressão na etapa anterior e 40\% dos dados são usados para testes na etapa atual. Em seguida, os valores de treino e testes são ajustados para 70\% e 30\%, respectivamente. Essa alteração é feita sucessivamente para os valores de 80\% e 20\% até os valores finais de 90\% e 10\%. Esse ciclo é repetido para todas as bases de dados e todas as versões dos modelos \textit{BERT}. A precisão dessas avaliações é verificada em termos de acurácia percentual. A acurácia média é então calculada para todas as avaliações geradas nos testes, permitindo avaliar o desempenho do algoritmo.

\subsection{Execução do algoritmo em diferentes bases e modelos}

Todo o ciclo de metodologia é executado múltiplas vezes para cada uma das bases de dados e das diferentes versões dos modelos, afim de obter a maior quantidade de dados possíveis para avaliação da acurácia do algoritmo no final em vários casos e observar em quais casos os resultados obtidos foram melhores.

\subsection{Otimização do algoritmo}

No decorres do desenvolvimento do presente trabalho, uma dificuldade enfrentada foi otimizar a execução do algoritmo para lidar com grandes volumes de dados de forma eficiente. Isso exigiu a implementação de estratégias avançadas de otimização, como a paralelização com \textit{multi-threading}, a programação dinâmica, uso de protocolos como \textit{Open MPI} e a execução dos modelos na \textit{GPU} com \textit{CUDA}.

A paralelização com \textit{multi-threading} permitiu que o algoritmo executasse múltiplas tarefas em paralelo, acelerando o processamento de grandes conjuntos de dados. A programação dinâmica otimizou o uso de recursos computacionais e reduziu a complexidade algorítmica, garantindo uma execução mais eficiente do algoritmo e menos repetições desnecessárias de trechos de código. A divisão de processos com protocolo \textit{MPI} permitiu que diferentes partes do algoritmo trocassem dados de forma assíncrona e coordenada, dividindo a carga de trabalho. Já a execução dos modelos na \textit{GPU} com \textit{CUDA} acelerou o processamento de dados, aproveitando o poder de processamento das \textit{GPU's} para lidar com grandes volumes de dados de forma eficiente.

Essas técnicas de otimização garantiram uma execução eficiente do algoritmo, permitindo lidar com grandes volumes de dados e garantir tempos de execução adequados. Antes desse processo de otimização, o tempo estimado de execução do algoritmo estava em cerca de 10 dias cada vez que era rodado. Após as otimizações do código, o algoritmo pode ser executado completamente em cerca de 40 minutos. A experiência técnica e o conhecimento especializado obtidos ao longo da graduação foram fundamentais para superar os desafios enfrentados durante o desenvolvimento do algoritmo e garantir sua eficácia e otimização nos tempos de execução.

\section{\textit{Datasets utilizados}}

Para este projeto de avaliação automática de respostas curtas, foram utilizadas três bases de dados distintas, cada uma em um idioma específico. A primeira base de dados consiste em um conjunto de questões de Biologia em português, a segunda base de dados é composta por questões de Literatura em espanhol e a terceira base de dados contém questões de Ciência da Computação em inglês.

A base de dados em português foi apresentada por \textcite{datasetPort} criada em 2020, contendo cerca de 15 questões e 23350 respostas de alunos.

Além disso, o conjunto de dados em espanhol apresentado por \textcite{datasetEsp} em 2023, contem cerca de 20 questões e 3770 respostas de alunos.

Por fim, a base de dados em inglês empregada foi apresentada por \textcite{datasetEng} em 2009, contando com 85 questões e 3645 respostas de alunos.

% \begin{figure}[h!]
% \includegraphics[width=\textwidth]{img/diagramasTcc.png}
% \caption{Diagrama geral das etapas do processo.}\label{figure:4}
% \end{figure}

% olhar depois:

% https://www.researchgate.net/publication/337412528_Avaliacao_Automatica_de_respostas_discursivas_curtas_baseado_em_tres_dimensoes_linguisticas

% https://revistas.unoeste.br/index.php/ce/article/view/4595

% https://physionet.org/content/radqa/1.0.0/

% https://ppgee.propesp.ufpa.br/ARQUIVOS/teses/tese_abntex_final.pdf

% dataset search google

\newpage
\chapter{PROPOSTA EXPERIMENTAL}

Como proposta experimental, será realizado a avalição das questões dissertativas através de um algoritmo que leva em consideração a métrica de pontuação proposta nos capitulos anteriores do texto, tanto com seus fatores em conjunto, quanto com eles individualmente, o resultado das avaliações realizadas pelo algoritmo será comparado aos valores de avaliação dos professores que estão presentes nas bases de dados específicadas anteriormente no capítulo de metodologia.

A proposta experimental pode ser dividida nos seguintes tópicos:
\begin{itemize}
	\item \textbf{Avaliação individual da presença de palavras chaves:} O algoritmo proposto deve avaliar se as palavras chaves determinadas para aquela questão, estão presentes no texto e pontuar com base na quantidade de palavras-chave encontradas. Tendo como hipótese que essa avaliação individualemte não fornecerá um resultado com alta precisão.
	\item \textbf{Avalição individual da frequência de palavras (SFD):} O algoritmo realizará a medição da frequência das palavras no texto da resposta padrão e da resposta fornecida pelo aluno, cada palavra terá sua frequência em ambos os textos e a diferença entre as duas será medida em porcentagem. Caso palavras diferentes apareçam no texto da respota do aluno, podemos levar em consideração inicialmente a possibilidade de descartá-las, outras possibilidades futuras também podem ser consideradas caso melhores abordagens sejam encontradas na literatura científica.
	\item \textbf{Avalição individual da sintaxe do texto:} Utilizando vetores de ordem de palavras o algoritmo deve realizar a medição das métricas de sintaxe do texto, primeiramente da resposta padrão, em sequência, das respotas dos alunos. O vetor de ordem de palavras da respota padrão será comparado com os vetores de ordem de palavras que representem as respostas dos alunos.
	\item \textbf{Avaliçao semantica através da distância de cossenos:} Na avaliação da distância de cossenos  os textos serão representados em vetores de palavras que representam a similaridade direcional, não a ordem como no item anterior, conforme os ângulos calculado entre os vetores diminui, sabemos que o grau de similaridade entre os textos aumenta. Dentro de diferentes técnicas de medição, a revisão bibliográfica demonstrou que essa é a mais precisa \cite{SemanticSimilarityBasedDescriptiveAnswerEvaluation}.
\end{itemize}

Além das avaliações individuais definidas acima, também serão avaliados os fatores em diferentes conjuntos para que os resultado obtidos através dessas combinções sejam comparados com a junção de todos os quatro fatores em uma só métrica, tendo como hipótese que as métricas com um conjunto de fatores combinados funcionarão melhor do que um fator individualmente e que a métrica de todos os fatores juntos terá o melhor resultado. 

Em todas as métricas combinadas, os diferentes fatores devem possuir pesos que representem o quão relevante cada elemento é para uma avalição da resposta. Com base nas avaliações individuais realizadas inicialmente, serão obtido quais dos fatores possuem uma proximidade maior com os valores das avalições dos professores disponíveis nas bases de dados, podemos, inicialmente, propor que quanto maior a proximidade dos valores individuais obtidos com as avaliações nas bases de dados, mais relevantes esse fator deve ser em seu peso. 

Com o objetivo de obter a melhor precisão possível, os pesos inicias serão modificados para valores com maior ou menor relevância, afim de que o resultado final obtido seja o mais próximo possível dos resultados das avaliações feitas pelos professores disponíveis nas bases de dados. Além disso, métricas como BLEU e ROUGE podem ser empregadas para avaliar a qualidade da pontuação em termos de sobreposição de palavras entre as respostas.

A proposta experimental busca não apenas avaliar a viabilidade do algoritmo, mas também otimizar sua precisão por meio da ponderação adequada dos fatores. A abordagem modular, combinada com a análise iterativa dos pesos, visa criar um sistema capaz de fornecer resultados de avaliação de respostas dissertativas com alta concordância em relação às avaliações humanas.

\section{Cronograma}
Na Tabela \ref{table:3} está estruturado o cronograma proposta para as fases que já concluídas e as que deverão ser concluídas afim de alcaçar os objetivos propostos.

\newcommand{\gray}{\cellcolor{black!25}}
\begin{table}[ht]
\label{tab:cronograma}
\resizebox{\textwidth}{!}{
\begin{tabular}{llllrll|llllll}
\hline
\multicolumn{1}{c}{ } &
  \multicolumn{6}{c|}{2023} &
  \multicolumn{6}{c}{2024} \\ \hline
\multicolumn{1}{|l|}{Atividade / Mês} &
  \multicolumn{1}{l|}{07} &
  \multicolumn{1}{l|}{08} &
  \multicolumn{1}{l|}{09} &
  \multicolumn{1}{r|}{10} &
  \multicolumn{1}{l|}{11} &
  12 &
  \multicolumn{1}{l|}{01} &
  \multicolumn{1}{l|}{02} &
  \multicolumn{1}{l|}{03} &
  \multicolumn{1}{l|}{04} &
  \multicolumn{1}{l|}{05} &
  \multicolumn{1}{l|}{06} \\ \hline
\multicolumn{1}{|l|}{Definição do tema e orientador} &
  \multicolumn{1}{l|}{\cellcolor[HTML]{656565}} &
  \multicolumn{1}{l|}{\cellcolor[HTML]{656565}} &
  \multicolumn{1}{l|}{} &
  \multicolumn{1}{r|}{} &
  \multicolumn{1}{l|}{} &
   &
  \multicolumn{1}{l|}{{\ul }} &
  \multicolumn{1}{l|}{} &
  \multicolumn{1}{l|}{} &
  \multicolumn{1}{l|}{} &
  \multicolumn{1}{l|}{} &
  \multicolumn{1}{l|}{} \\ \hline
\multicolumn{1}{|l|}{Revisão bibliográfica} &
  \multicolumn{1}{l|}{} &
  \multicolumn{1}{l|}{} &
  \multicolumn{1}{l|}{\cellcolor[HTML]{656565}} &
  \multicolumn{1}{l|}{\cellcolor[HTML]{656565}} &
  \multicolumn{1}{l|}{} &
   &
  \multicolumn{1}{l|}{{\ul }} &
  \multicolumn{1}{l|}{} &
  \multicolumn{1}{l|}{} &
  \multicolumn{1}{l|}{} &
  \multicolumn{1}{l|}{} &
  \multicolumn{1}{l|}{} \\ \hline
\multicolumn{1}{|l|}{Planejamento da metodologia} &
  \multicolumn{1}{l|}{} &
  \multicolumn{1}{l|}{} &
  \multicolumn{1}{l|}{} &
  \multicolumn{1}{l|}{\cellcolor[HTML]{656565}} &
  \multicolumn{1}{l|}{\cellcolor[HTML]{656565}} &
   &
  \multicolumn{1}{l|}{{\ul }} &
  \multicolumn{1}{l|}{} &
  \multicolumn{1}{l|}{} &
  \multicolumn{1}{l|}{} &
  \multicolumn{1}{l|}{} &
  \multicolumn{1}{l|}{} \\ \hline
\multicolumn{1}{|l|}{Finalização do relatório parcial} &
  \multicolumn{1}{l|}{} &
  \multicolumn{1}{l|}{} &
  \multicolumn{1}{l|}{} &
  \multicolumn{1}{r|}{} &
  \multicolumn{1}{l|}{} &
  \cellcolor[HTML]{656565} &
  \multicolumn{1}{l|}{{\ul }} &
  \multicolumn{1}{l|}{} &
  \multicolumn{1}{l|}{} &
  \multicolumn{1}{l|}{} &
  \multicolumn{1}{l|}{} &
  \multicolumn{1}{l|}{} \\ \hline
\multicolumn{1}{|l|}{Aprofundamento nos conceitos de NLP} &
  \multicolumn{1}{l|}{} &
  \multicolumn{1}{l|}{} &
  \multicolumn{1}{l|}{} &
  \multicolumn{1}{r|}{} &
  \multicolumn{1}{l|}{} &
   &
  \multicolumn{1}{l|}{\cellcolor[HTML]{656565}{\ul }} &
  \multicolumn{1}{l|}{} &
  \multicolumn{1}{l|}{} &
  \multicolumn{1}{l|}{} &
  \multicolumn{1}{l|}{} &
  \multicolumn{1}{l|}{} \\ \hline
\multicolumn{1}{|l|}{\begin{tabular}[c]{@{}l@{}}Experimentação do algoritmo para\\ avaliação da relevância dos fatores\end{tabular}} &
  \multicolumn{1}{l|}{} &
  \multicolumn{1}{l|}{} &
  \multicolumn{1}{l|}{} &
  \multicolumn{1}{r|}{} &
  \multicolumn{1}{l|}{} &
   &
  \multicolumn{1}{l|}{{\ul }} &
  \multicolumn{1}{l|}{\cellcolor[HTML]{656565}} &
  \multicolumn{1}{l|}{} &
  \multicolumn{1}{l|}{} &
  \multicolumn{1}{l|}{} &
  \multicolumn{1}{l|}{} \\ \hline
\multicolumn{1}{|l|}{\begin{tabular}[c]{@{}l@{}}Experimentação para aprimoramento\\ dos valores dos pesos na métrica\end{tabular}} &
  \multicolumn{1}{l|}{} &
  \multicolumn{1}{l|}{} &
  \multicolumn{1}{l|}{} &
  \multicolumn{1}{r|}{} &
  \multicolumn{1}{l|}{} &
   &
  \multicolumn{1}{l|}{{\ul }} &
  \multicolumn{1}{l|}{} &
  \multicolumn{1}{l|}{\cellcolor[HTML]{656565}} &
  \multicolumn{1}{l|}{} &
  \multicolumn{1}{l|}{} &
  \multicolumn{1}{l|}{} \\ \hline
\multicolumn{1}{|l|}{\begin{tabular}[c]{@{}l@{}}Validação dos resultados obtidos com\\ os dados de avaliações de professores\end{tabular}} &
  \multicolumn{1}{l|}{} &
  \multicolumn{1}{l|}{} &
  \multicolumn{1}{l|}{} &
  \multicolumn{1}{r|}{} &
  \multicolumn{1}{l|}{} &
   &
  \multicolumn{1}{l|}{{\ul }} &
  \multicolumn{1}{l|}{} &
  \multicolumn{1}{l|}{} &
  \multicolumn{1}{l|}{\cellcolor[HTML]{656565}} &
  \multicolumn{1}{l|}{} &
  \multicolumn{1}{l|}{} \\ \hline
\multicolumn{1}{|l|}{\begin{tabular}[c]{@{}l@{}}Adição dos fatores de parafraseamento\\ para implementar a segunda métrica\end{tabular}} &
  \multicolumn{1}{l|}{} &
  \multicolumn{1}{l|}{} &
  \multicolumn{1}{l|}{} &
  \multicolumn{1}{r|}{} &
  \multicolumn{1}{l|}{} &
   &
  \multicolumn{1}{l|}{{\ul }} &
  \multicolumn{1}{l|}{} &
  \multicolumn{1}{l|}{} &
  \multicolumn{1}{l|}{} &
  \multicolumn{1}{l|}{\cellcolor[HTML]{656565}} &
  \multicolumn{1}{l|}{} \\ \hline
\multicolumn{1}{|l|}{\begin{tabular}[c]{@{}l@{}}Ajustes finais para refinamento do \\ algoritmo e finalização do projeto \\ (Relatório final e material do INOVA FEI)\end{tabular}} &
  \multicolumn{1}{l|}{} &
  \multicolumn{1}{l|}{} &
  \multicolumn{1}{l|}{} &
  \multicolumn{1}{r|}{} &
  \multicolumn{1}{l|}{} &
   &
  \multicolumn{1}{l|}{{\ul }} &
  \multicolumn{1}{l|}{} &
  \multicolumn{1}{l|}{} &
  \multicolumn{1}{l|}{} &
  \multicolumn{1}{l|}{} &
  \multicolumn{1}{l|}{\cellcolor[HTML]{656565}} \\ \hline
\end{tabular}
}
\caption{Cronograma de execução do projeto}
\label{table:3}
\end{table}

\newpage
\chapter{CONCLUSÃO}
Levando em consideração tudo o que foi exposto ao longo de todos os capítulos, pode-se concluir, deste trabalho, que propôs uma abordagem para automatizar o processo de avaliação de respostas dissertativas, que os resultados obtidos apontam para uma direção interessante e promissora. Foi possível gerar uma grande quantidade de avaliações com resultados próximos aos de avaliações de professores. A métrica proposta, que levou em consideração a frequência de termos, distância de Levenshtein e similaridade semântica obtidas através de modelos BERT, visando simplificar o processo de correção manual para possibilitar a análise do desempenho dos alunos mais rapidamente, demonstrou seu funcionamento com resultados razoáveis e compreensíveis dentro dos limites do que este trabalho teve como proposta. Ao utilizar os \textit{DataSets} de diferentes temáticas e idiomas, foi possível estimar pesos para os fatores que compõem a métrica de avaliação. Com eles, gerar avaliações automáticas e coletar os dados de acurácia da métrica, que apresentaram um resultado que pode ser considerado satisfatório para os fatores presentes e a quantidade de dados disponibilizados, além do tempo disponível para conclusão deste trabalho.

Algumas limitações foram observadas nos resultados, em que no caso de alguns \textit{datasets}, foi possível ver uma quantidade grande de erros em faixas percentuais muito altas. Avaliações com uma diferença de valor alta quando comparadas com as avaliações dos professores. Uma explicação para muitos desses casos é a falta de compreensão de mais fatores relativos ao texto que o algoritmo ainda não possui a capacidade de avaliar.

Tendo isso em vista, como possibilidades de trabalhos futuros para melhoria dos resultados, algumas abordagens podem ser consideradas, como por exemplo:

\begin{itemize}
    \item \textbf{Inclusão de mais fatores no algoritmo}: Avaliar a inclusão de mais fatores ou características no algoritmo pode ajudar a capturar melhor a complexidade das respostas dissertativas em diferentes idiomas. Fatores como identificação de paráfrases e mais características da semântica do texto podem ajudar a alcançar uma métrica mais acertiva.
    \item \textbf{Uso de outros modelos ou redes neurais}: Comparar o desempenho do algoritmo de regressão linear com outros modelos ou redes neurais pode fornecer boas considerações sobre qual abordagem é mais eficaz para determinação dos pesos.
\end{itemize}

Essas abordagens podem ajudar a aprimorar a precisão e a generalização do algoritmo para qualquer temática, tornando-o mais robusto e eficaz na avaliação de respostas dissertativas em português, inglês, espanhol ou até outros idiomas, eventualmente.

Em relação às questões de pesquisa levantadas para este trabalho, podemos considerar que:

\begin{enumerate}
    \item Os parâmetros extraídos foram capazes de avaliar uma grande quantidade de respostas, embora com a complementação de mais fatores e mais dados para treinamento com outros modelos, os resultados podem ser melhorados.
    \item Esses parâmetros foram capazes de definir uma pontuação que funcionou de maneira geral para avaliação de respostas dissertativas, embora houvessem limitações na acurácia da métrica.
    \item Os fatores devem ser ponderados com pesos, que podem ser otimizados com um volume maior de dados ou outros algoritmos de inteligência artificial, como redes neurais, para determiná-los.
    \item Podemos constatar que a acurácia da métrica variou consideravelmente comparando os idiomas e também o volume de dados para treinamento e teste. Isso fez com que os valores de acurácia do algoritmo fossem mais próximos quando treinados com os \textit{datasets} em inglês e espanhol.
\end{enumerate}

Em última análise, este trabalho almejou não apenas uma solução para as limitações atuais na avaliação de respostas dissertativas, mas também contribuir para a determinação de quais parâmetros são relevantes em uma métrica para esse problema de pesquisa. Com a determinação mais detalhada e profunda desses parâmetros em trabalhos futuros, abre-se a possibilidade do surgimento de uma contribuição significativa para os processos educacionais, proporcionando aos professores uma ferramenta valiosa para avaliação e acompanhamento do progresso dos alunos.

\printbibliography

\end{document}
