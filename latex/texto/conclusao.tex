\chapter{CONCLUSÃO}

Diante da diversidade de métodos de avaliação empregados nas instituições de ensino contemporâneas, as questões dissertativas destacam-se por proporcionarem uma compreensão mais profunda da linha de raciocínio dos alunos. Embora a precisão seja uma vantagem desse método, sua efetividade muitas vezes é prejudicada pelo tempo exigido para leitura e análise por parte dos professores. A preferência por questões objetivas, embora economize tempo, diminui a precisão da avaliação.

Este trabalho propõe uma abordagem para amenizar esse problema, buscando automatizar o processo de avaliação de respostas dissertativas. A métrica proposta leva em consideração a presença de palavras-chave, a frequência de sentidos de palavras, a sintaxe do texto e a distância semântica através do cosseno, visando simplificar o processo de correção manual, possibilitando analisar o desempenho dos alunos.

Ao utilizar o DataSet ASQA e dados de respostas para questões do ENADE, este trabalho planeja realizar experimentos para validar a eficácia da abordagem proposta. Avaliando quais fatores seriam relevantes para compor a métrica de avaliação. Esta pesquisa busca não apenas simplificar o processo de avaliação, mas também abrir a possibilidade de uma integração futura entre plataformas educacionais e a técnica proposta, eventualmente, proporcionando uma solução prática para a área da educação.

As questões de pesquisa levantadas durante a revisão bibliográfica e estabelecidas nos objetivos, relacionadas à paráfrase, grau de similaridade semântica e parâmetros relevantes para pontuação, serão abordadas na fase experimental, visando contribuir não apenas com avanços teóricos, mas também com aplicações práticas e melhorias efetivas na avaliação automática de respostas dissertativas.

Em última análise, este trabalho almeja não apenas uma solução para as limitações atuais na avaliação de respostas dissertativas, mas também contribuir na determinação de quais parâmetros são relevantes em uma métrica para esse problema de pesquisa. Ademais, o resultado prático do presente trabalho, caso os objetivos sejam alcançados, deverá oferecer uma contribuição significativa os processos educacionais, proporcionando aos professores uma ferramenta valiosa para avaliação e acompanhamento do progresso dos alunos.