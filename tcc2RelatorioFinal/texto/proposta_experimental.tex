\chapter{PROPOSTA EXPERIMENTAL}

Como proposta experimental, será realizado a avalição das questões dissertativas através de um algoritmo que leva em consideração a métrica de pontuação proposta nos capitulos anteriores do texto, tanto com seus fatores em conjunto, quanto com eles individualmente, o resultado das avaliações realizadas pelo algoritmo será comparado aos valores de avaliação dos professores que estão presentes nas bases de dados específicadas anteriormente no capítulo de metodologia.

A proposta experimental pode ser dividida nos seguintes tópicos:
\begin{itemize}
	\item \textbf{Avaliação individual da presença de palavras chaves:} O algoritmo proposto deve avaliar se as palavras chaves determinadas para aquela questão, estão presentes no texto e pontuar com base na quantidade de palavras-chave encontradas. Tendo como hipótese que essa avaliação individualemte não fornecerá um resultado com alta precisão.
	\item \textbf{Avalição individual da frequência de palavras (SFD):} O algoritmo realizará a medição da frequência das palavras no texto da resposta padrão e da resposta fornecida pelo aluno, cada palavra terá sua frequência em ambos os textos e a diferença entre as duas será medida em porcentagem. Caso palavras diferentes apareçam no texto da respota do aluno, podemos levar em consideração inicialmente a possibilidade de descartá-las, outras possibilidades futuras também podem ser consideradas caso melhores abordagens sejam encontradas na literatura científica.
	\item \textbf{Avalição individual da sintaxe do texto:} Utilizando vetores de ordem de palavras o algoritmo deve realizar a medição das métricas de sintaxe do texto, primeiramente da resposta padrão, em sequência, das respotas dos alunos. O vetor de ordem de palavras da respota padrão será comparado com os vetores de ordem de palavras que representem as respostas dos alunos.
	\item \textbf{Avaliçao semantica através da distância de cossenos:} Na avaliação da distância de cossenos  os textos serão representados em vetores de palavras que representam a similaridade direcional, não a ordem como no item anterior, conforme os ângulos calculado entre os vetores diminui, sabemos que o grau de similaridade entre os textos aumenta. Dentro de diferentes técnicas de medição, a revisão bibliográfica demonstrou que essa é a mais precisa \cite{SemanticSimilarityBasedDescriptiveAnswerEvaluation}.
\end{itemize}

Além das avaliações individuais definidas acima, também serão avaliados os fatores em diferentes conjuntos para que os resultado obtidos através dessas combinções sejam comparados com a junção de todos os quatro fatores em uma só métrica, tendo como hipótese que as métricas com um conjunto de fatores combinados funcionarão melhor do que um fator individualmente e que a métrica de todos os fatores juntos terá o melhor resultado. 

Em todas as métricas combinadas, os diferentes fatores devem possuir pesos que representem o quão relevante cada elemento é para uma avalição da resposta. Com base nas avaliações individuais realizadas inicialmente, serão obtido quais dos fatores possuem uma proximidade maior com os valores das avalições dos professores disponíveis nas bases de dados, podemos, inicialmente, propor que quanto maior a proximidade dos valores individuais obtidos com as avaliações nas bases de dados, mais relevantes esse fator deve ser em seu peso. 

Com o objetivo de obter a melhor precisão possível, os pesos inicias serão modificados para valores com maior ou menor relevância, afim de que o resultado final obtido seja o mais próximo possível dos resultados das avaliações feitas pelos professores disponíveis nas bases de dados. Além disso, métricas como BLEU e ROUGE podem ser empregadas para avaliar a qualidade da pontuação em termos de sobreposição de palavras entre as respostas.

A proposta experimental busca não apenas avaliar a viabilidade do algoritmo, mas também otimizar sua precisão por meio da ponderação adequada dos fatores. A abordagem modular, combinada com a análise iterativa dos pesos, visa criar um sistema capaz de fornecer resultados de avaliação de respostas dissertativas com alta concordância em relação às avaliações humanas.

\newpage