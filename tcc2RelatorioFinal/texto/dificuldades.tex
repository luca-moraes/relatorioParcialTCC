\chapter{DIFICULDADES E SOLUÇÕES}

Durante o desenvolvimento do algoritmo para a avaliação automática de respostas dissertativas, várias dificuldades foram enfrentadas, cada uma exigindo soluções específicas para garantir o sucesso do projeto, nessa secção serão abordadas as dificuldades encontradas e as soluções escolhidas para resoolve-las.

\section{Seleção e Implementação de Métricas Adequadas}

Um desafio significativo enfrentado foi a seleção e implementação de métricas apropriadas para avaliar a qualidade das respostas dissertativas. Foi necessário realizar uma pesquisa extensiva de técnicas avançadas de processamento de linguagem natural e inteligência artificial, adaptando-as ao contexto específico da avaliação educacional. Colaboração com especialistas em aprendizado de máquina e linguística computacional foi crucial para identificar e implementar as métricas mais relevantes de maneira eficiente.

\subsection{Regressão Linear}

A regressão linear foi uma das métricas escolhidas para avaliar a relação entre as avaliações realizadas pelo algoritmo e as avaliações dos professores. Utilizando técnicas estatísticas avançadas, pudemos modelar essa relação e quantificar a precisão do algoritmo em relação às avaliações humanas.

\subsection{Mean Squared Error (MSE) e Mean Absolute Error (MAE)}

O MSE e o MAE foram utilizados para avaliar a precisão das previsões do modelo de regressão linear. O MSE quantifica o erro médio quadrático entre as previsões do modelo e os valores reais, enquanto o MAE calcula o erro médio absoluto. Essas métricas foram essenciais para avaliar a eficácia do modelo de regressão linear na predição das avaliações das respostas dissertativas.

\subsection{Erro Médio Geral e Faixas de Erro Percentual}

Além disso, calculamos o erro médio geral entre as avaliações realizadas pelo algoritmo e as avaliações dos professores, bem como as faixas de erro percentual para entender melhor a distribuição dos erros.

\section{Otimização da Execução do Algoritmo}

Uma dificuldade adicional foi otimizar a execução do algoritmo para lidar com grandes volumes de dados de forma eficiente. Isso exigiu a implementação de estratégias avançadas de otimização, incluindo:

\subsection{Paralelização com Multi-Thread}

Para lidar com a computação simultânea de tarefas, implementamos paralelização com multi-threading, permitindo que o algoritmo execute múltiplas tarefas em paralelo, o que acelera significativamente o processamento de grandes conjuntos de dados, pois paralelizamos o calculo dos tres fatores dentro dos textos das respostas.

\subsection{Programação Dinâmica}

Utilizamos técnicas de programação dinâmica para otimizar o uso de recursos computacionais e reduzir a complexidade algorítmica, garantindo uma execução mais eficiente do algoritmo e menos repetições desnecessárias de trechos de código.

\subsection{Mensageria com MPI}

Para comunicação eficiente entre processos em sistemas distribuídos, implementamos mensageria com MPI (Message Passing Interface), permitindo que diferentes partes do algoritmo troquem dados de forma assíncrona e coordenada, dividindo a carga de trabalho.

\subsection{Execução dos Modelos na GPU com CUDA}

Para acelerar o processamento de dados, utilizamos a GPU (Graphics Processing Unit) com CUDA (Compute Unified Device Architecture) para executar os modelos de forma paralela, aproveitando o poder de processamento massivo das GPUs para lidar com grandes volumes de dados de forma eficiente.

Essas técnicas de otimização garantiram uma execução eficiente do algoritmo, permitindo lidar com grandes volumes de dados e garantir tempos de execução adequados para a avaliação que retirava os fatores dos textos das respostas.

Essas dificuldades foram superadas com a experiência técnica e o conhecimento especializado obtido ao longo da graduação, tudo isso foi fundamental para superar os desafios enfrentados durante o desenvolvimento do algoritmo e garantir sua eficácia e otimização nos tempos de execução.

\newpage